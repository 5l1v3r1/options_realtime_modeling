\documentclass{article}
\usepackage[utf8]{inputenc}
\usepackage{graphicx}
\usepackage{flafter}
\usepackage{float}
\graphicspath{ {./images/} }

\title{Real-Time Options Pricing Model}
\author{Jaisal Friedman}
\date{December 16 2018}
\begin{document}
\maketitle


\begin{abstract}
    This paper explores the Black-Scholes-Merton options pricing model, derives a predictive extension model, and visualizes both models in comparison to real-time pricing options pricing. The paper also explores various methodologies of calculating historical volatility. A portfolio of 5 U.S. Market Stocks and 1 index fund was taken as example for the project. The model was limited to visual analysis from real-time simulations as further explained in the extensions section. 
\end{abstract}


\section{Introduction}
\begin{flushleft} The impact of the Black-Scholes-Merton model on modern financial markets drove the paper's initial fascination. The paper intends to visually model the Black-Scholes-Merton Model, a predictive extension model, and the real-time options spread for a better understanding of the options market. This project first seeks to explore the dynamics of the Black-Scholes-Merton and its derivation. The paper then introduces a predictive extension to which a visual analysis is displayed in real-time for one to decipher its effectiveness. Four methodologies for the historical volatility calculation were explored and compared in real-time between the 6 example assets. The Black-Scholes-Merton calculation for real-time assets with historical data was then modeled. The extended predictive model was derived for real-time assets. And lastly, the F-B-S, predictive model, and real-time options pricing was visually displayed and analyzed using 3D volatility surfaces. The paper, itself, will begin with the methodology: where the historical volatility calculations, assumptions of the Black-Scholes-Merton model, and the predictive extension will be discussed. Then introduce the implementation: where the real-time numerical simulation, 3D visualization, and discussion of results are presented. Lastly, it will conclude with a short conclusion, explanation of the software development, limitations, and possible extensions and improvements of the model.
\end{flushleft}


\section{Methodology}

\subsection{Introduction to the Black-Scholes-Merton Model}
\begin{flushleft}
The Black-Scholes model was initially published in 1973. It provided the first deterministic pricing model for the options market. This led to a boom in the options financial markets that has continued till present day. In 1976, Merton published his extension of the Black-Scholes model which included the effect of dividend payouts. Today, the Black-Scholes-Merton model is used to determine the Greeks and Implied Volatility in the financial options market. Its impact on efficiently pricing the options market as a benchmark model holds strong even some 40 years later. For this reason, this paper starts its extension and comparative visualization with such a model. 
\end{flushleft}
\subsection{Black-Scholes-Merton Equation}
\begin{flushleft}
$C_t(s,t) = S \cdot e^{-qt}  \cdot N (d_1) - X\cdot e^{-rt} \cdot N(d_2)$
\newline\newline
$P_t(S,t) = -e^{-qt}  \cdot  S \cdot N (-d_1) + X\cdot e^{-rt} \cdot N(-d_2)$
\newline\newline
where: 
\newline\newline
$d_1 = \frac {ln(\frac{S}{X}) + (r-q+ \frac{\sigma^2}{2}) \cdot t}{\sigma \cdot \sqrt {t}}$
\newline\newline
$d_2 = d_1 - \sigma \cdot \sqrt{t}$
\newline\newline
$N(x) = \int_{-\infty}^{x} {\frac {1}{\sqrt{2\pi}\cdot \sigma^2}\cdot e ^{\frac{-(u-\mu)^2}{2\sigma^2}} du},$
where $\sigma^2 = 1, \mu = 0$
\end{flushleft}

\subsubsection {Notable Assumptions}
\begin{flushleft}
\begin{enumerate}
\item The cumulative distribution function N(x) follows market probabilities. In fact, there are unpredictable Black Swan events which severely skew market movements. Fat Tail distributions are more accurate for financial markets. 
\item  Market speculation has no effect on the pricing of options.
\end{enumerate}

\end{flushleft}

\subsubsection {Historical Volatility $\sigma$}
\begin{flushleft} Historical volatility in the Black-Scholes-Merton $\sigma$ can be calculated by different methods. Classically, $\sigma$ is calculated by either the standard deviation of the log returns or the percent change of close-to-close asset pricing for the past 30 days normalized to the last trading year. This methodology utilizes the close-to-close historical pricing. The log returns is argued to be a more mathematically sound indicator than percent change for $\sigma$. 
\newline \newline
$hv_{percent} = \sqrt{\frac{1}{N-1} \sum_{i=1}^N (x_i - \overline{x})^2},$ where $x=\frac{s(t)_{close} - s(t-1)_{close}}{s(t-1)_{close}} $
\newline\newline
$hv_{log} = \sqrt{\frac{1}{N-1} \sum_{i=1}^N (x_i - \overline{x})^2},$ where $x=log{\frac{S^t_{close}}{ S^{t-1}_{close}}}$
\end{flushleft}
\begin{flushleft} However, Historical volatility in the Black-Scholes-Merton $\sigma$ can also be calculated by intraday methods. This paper presents two possible extensions. The first is self-derived. It utilizes both the open-close and high-low pricing to return a weighted intraday historical volatility. The second is the highly referenced Parkinson Historical Intraday volatility calculation. Both methodologies utilizes the intraday historical pricing and provide a different outlook at on $\sigma$. 
\newline \newline
$hv_{w_ext} = \sqrt{\frac{1}{N-1} \sum_{i=1}^N (x_i - \overline{x})^2},$ where $x=log{\frac{S^t_{high}}{ S^{t-1}_{low}}} \cdot w + log{\frac{S^t_{high}}{ S^{t-1}_{low}}} \cdot (1-w)$
\newline\newline
$hv_{w_park} = \sqrt{\frac{1}{N} \sum_{i=1}^N \frac{1}{4\ln(2)}(x_i )^2},$ where $ x=log{\frac{S^t_{high}}{ S^{t-1}_{low}}} \cdot $
\end{flushleft}

\subsection{Predictive Extension}
\begin{flushleft}
The predictive extension presented in this paper builds off the Black-Scholes-Merton model. The model attempted to scale the B-S-M model by sigmoid function encapsulating current financial market health and beta exposure of the underlying asset to the market conditions.
\end{flushleft}

\subsubsection{Equation}
$M(c_1, c_2, dd, \beta, t) = \frac{x}{1+ (|x|)} + 1,$ where $x = \frac{c_1\beta\sqrt{t}}{dd} - c_2$
\newline\newline
$C_{ext}(s,t) = S \cdot e^{-qt}  \cdot N (d_1) - X\cdot e^{-rt} \cdot N(d_2) \cdot M(c_1, c_2,dd, \beta, t)$
\newline \newline
$P_{ext}(S,t) = (-e^{-qt}  \cdot  S \cdot N (-d_1) + X\cdot e^{-rt} \cdot N(-d_2)) \cdot \frac{1}{M(c_1, c_2,dd, \beta, t)}$
\newline \newline
where $c_1, c_2$ = constants, dd = defaulted debt and $\beta$ = risk exposure of an asset to the market
\newline
\subsubsection {Notable Assumptions}
\begin{flushleft}  
\begin{enumerate}
\item  Speculation about the current market conditions has an impact on the pricing of long-dated options. 
\item The $\beta$ exposure of an underlying asset to the market affects the pricing of certain assets over assets with a lower $\beta$. 
\item The time of the options has a square-root proportionality with the pricing of options under certain market conditions. Longer-dated options will be square-root proportionally more affected by overall market speculation. 
\item The dd, defaulted debt, indicator can be effectively represent the overall financial market health. 
\item The M function follows a Sigmoid proportionality with the Black-Scholes pricing model centered by $c_1, c_2$. 
\end{enumerate}
\end{flushleft}

\subsection {Real-Time Pricing}
\begin{flushleft} 
The real-time pricing involved a substantial amount of software development. This was deemed worthy as the model's visual analysis is better implemented in real-time. 
\end{flushleft}
\subsubsection {Assets Chosen}
\begin{itemize}
\item MSFT - Microsoft
\item BA - Boeing
\item V - Visa 
\item AAPL - Apple
\item GOOGL - Alphabet Class A
\item SPY -  SPDR S&P 500 ETF Trust
\end{itemize}
\begin{flushleft} 
These assets were chosen directly from my personal portfolio as they held the most interest to me. In the code, one can select their own portfolio assets to visualize. 
\end{flushleft}
\subsubsection {API Interactions}
\begin{enumerate}
\item FRED - Federal Reserve Economic Data
\item IEX Exchange - Real-time Stock Data 
\item Tradier - Real-time Options Data
\item Plotly - Interactive 3D Visualizations
\end{enumerate}

\newpage
\section{Implementation}
\subsection{Numerical Simulation}
\begin{flushleft}
The implementation of the methodology was carried out in python. We built a working self-balancing structure which could take the following parameters, we then ran the simulation across multiple $\theta_{max}$ and $\omega$ values, tracked the resultant quantities (shown below), and graphed them (displayed in the results section): \newline
Parameter Sweeps: 
\begin{itemize}
    \item $\theta_{max}$
    \begin{itemize}
        \item $\ 0 \leq \theta \leq \theta_{max}$ rad
        \item $ \frac{\pi}{64} \leq \theta_{max} \leq \frac{\pi}{16} $ rad
        \item Constant: $\omega$ = $\frac{\pi}{100} \frac{\theta}{s}$ rad/s
    \end{itemize}
    \item $\omega$ sweep
    \begin{itemize}
        \item $\ \frac{\pi}{100} \leq \omega \leq \frac{\pi}{200}$ rad/s
        \item Constant: $\theta_{max}$ = $\pi/64$ rad
    \end{itemize}
\end{itemize}
Our simulation used the following constants: 
\begin{itemize}
    \item Globals
    \begin{itemize}
        \item Mass = 1 kg
        \item Spring Constant = 15000 N/m
        \item Dash-pot Constant = 600 N/m
        \item Gravity = 9.8 $m/s^2$
        \item Maximal Floor Strength = 10000 N
    \end{itemize}
    \item Structure
    \begin{itemize}
        \item Width = 2 m
        \item Strength = 3 N
        \item Levels = 5 dimensionless
    \end{itemize}
    \item Simulate
    \begin{itemize}
        \item dt = $1\cdot 10^{-4}$ s
        \item $t_{end}$ = 10 s
    \end{itemize}
\end{itemize}
We tracked the following resultant quantities in each simulation: 
\begin{itemize}
    \item Horizontal Center of Mass (m)
    \item Horizontal Center of Velocity (m/s)
    \item Sum of Vertical Displacement (m)
\end{itemize} 
\end{flushleft}
\subsection{Results}
\begin{flushleft}
1. \textbf{Parameter Sweep} over $theta_max$ and $\omega$ were run.\newline 
Figures 3,4,and 5 are the results for the parameter sweep of $theta_max$ and Figures 6,7,and 8 are the results for the parameter sweep of $\omega$. \newline
2. \textbf{Vertical Displacement per level} was graphed over a series of simulations to show perturbations to the structure. See Figure 9 in Appendix under Figures 

\textbf{These can be seen at in the Appendix section under Figures}
\end{flushleft}

\subsection{Discussion}


\begin{flushleft}{The simulation was carried out over a number of parameter combinations.  We altered the maximum angle of platform rotation and speed of platform rotation.  Our observations recorded the resulting effects of the parameter changes. From the Figures in the Appendix, we can see the instances where the figure becomes unstable at certain $\theta and \omega$ values. 
While the choice of parameters determines whether or not the entire structure will collapse, an interesting pattern is observed during stable conditions.  Stability is highest is the bottom structure and lowest in the second level, however, the trend then switches to increasing with height. This is due to the efforts of the level below contributing to an  task of compensation by the structure above.} 
\end{flushleft}


\section{Conclusion}
\subsection{Improvements}
\begin{flushleft} 
\begin{itemize}
  \item \textbf{Horizontal Gravity} \newline
  \tabHorizontal Horizontal Gravity is not accounted for in the model. This is a known area of improvement that the model was not able to account for. This resulted in a lesser horizontal velocity due to a lesser horizontal force when the structure was rotated. This would arguably help the structures ability to self-balance. This feature would be easy to implement by splitting the components of gravity. Unfortunately, the model would break if the x component of gravity was not set to 0.
  \item \textbf{Structural Tension across levels} \newline
  \tabHorizontal Structural tension was not transferred across all levels. This means the bottom levels did not bear the weight of the upper levels and thus did not bear the force of their position. This is also a feature, the model should include to obtain real-world accuracy. We built the model to be only one level; when we decided to extend the height of the model. We did not have the time to implement this structural code change to the force calculations and thus to our model. 
  \item \textbf{Reaction when right foot node > width} \newline
  \tabHorizontal When the right foot node reaches a y value greater than height, the structure can't react. This is because the right leg is set by $width\cdot sin(\theta)$. This is one of the reasons we bound the parameter sweep for $\omega$ and $\theta_{max}$ as we did. 
\end{itemize}
\end{flushleft}

\subsection{Extensions}
\begin{flushleft} 
\begin{itemize}
  \item \textbf{Fixing the Improvements} \newline
  \tabHorizontal Fixing the listed Improvements in the above section would clarify a more stable and real-world applicable parameter sweep. This could be used to distinguish the impact and limits of a hydraulic system on a basic building structure. Our model could be translated to a simplified model of a buildings structure via dimensional analysis.  
  \item \textbf{3D-Rendering} \newline
  \tabHorizontal Our model could easily be extended to include 3 dimensions. This would improve the visual appeal and real-world application of the model. Simulations with rotations around specific axes could also be distinguished and analyzed. This would be pertinent to using our structure to model any real-world self-balancing hydraulic system. 
  \item \textbf{Physical Therapy and Core Stability Modelling} \newline
  \tabHorizontal This project started with the fascination of the human body and its ability to self balance on tilted-platforms. This model could be extended to analyze a mechanism such as core strength by cutting the model down to 1 level, adding feet as the stable ground point, adding a head mass, and adding hands for balancing. This would be a difficult, but fascinating extension. 
\end{itemize}
\end{flushleft}


\section{Appendix}
\subsubsection{Code}
\begin{flushleft}
 The project was done in python & jupyter hub. The link to the github is here: https://github.com/jaisal1024/options_realtime_modeling
\end{flushleft}
\subsection{Figures}
\begin{figure}[h!]
\includegraphics[width =\textwidth]{images/turn.png}
\caption{Live Modelling of a Rotated Structure}
\centering
\end{figure}
\begin{figure}[h!]
\includegraphics[width =\textwidth]{images/theta_xcm.png}
\caption{Parameter Sweep for $\theta_{max}$}
\centering
\end{figure}
\begin{figure}[h!]
\includegraphics[width =\textwidth]{images/theta_vcm.png}
\caption{Parameter Sweep for $\theta_{max}$}
\centering
\end{figure}
\begin{figure}[h!]
\includegraphics[width =\textwidth]{images/theta_y.png}
\caption{Parameter Sweep for $\theta_{max}$}
\centering
\end{figure}
\begin{figure}[h!]
\includegraphics[width =\textwidth]{images/speedrot_xcm.png}
\caption{Parameter Sweep for $\omega$}
\centering
\end{figure}
\begin{figure}[h!]
\includegraphics[width =\textwidth]{images/speedrot_vcm.png}
\caption{Parameter Sweep for $\omega$}
\centering
\end{figure}
\newpage \newpage
\begin{figure}[h!]
\includegraphics[width =\textwidth]{images/speedrot_y.png}
\caption{Parameter Sweep for $\omega$}
\centering
\end{figure}
\begin{figure}[h!]
\includegraphics[width =\textwidth]{images/download-9.png}
\caption{Vertical Displacements over Levels of Structure}
\centering
\end{figure}

\end{document}
